\chapter{Problem statement}

% ~~~~~~~~~~~~~~~~~~~~~~~~~~~~~~~~~~~~~~~~~~~~~~~~~~~~~~~~~~~~~~~~~~~~~~~~~~~~~~~~~~~~~~~~~~~~~~~~~~~~~~~~~~~
\section{Scheduling}

\begin{itemize}
    \item General definition of \ac{rcpsp}

    \item Describe the $PSm \;|\; intree \;|\; \sum_{j} w_j T_j$ problem.
        Schema defined by \citet{BRUCKER1999}.
    \begin{itemize}
        \item \emph{Project} consisting of \emph{jobs} $1, \dots, n \in J$.
        \begin{itemize}
            % Jobs
            \item Each job $j$ has an execution \emph{duration} $\duration{j}$ - the amount of time needed to process the job $j$.
                \emph{Preemption} of jobs is not allowed - the execution of an job cannot be interrupted after its start.

            % Precedences
            \item Execution order of jobs is constrained with \emph{precedence constraints}.
                A precedence constraint between jobs $i$ and $j$, denoted as $\precedence{i}{j}$ or $(i, j)$,
                describes that processing of job $j$ can start only after the processing of job $i$ was completed.

                We denote the set of all jobs as $\Jobs$ and the set of all precedences as $\Precedences$.
                We then define the \emph{precedence graph} $G = (\Jobs, \Precedences)$.

                The precedence graph is an \emph{inforest}. An inforest is a directed acyclic graph
                where each vertex has at most one successor. A connected subgraph of an inforest
                is an \emph{intree}.

            % Tardiness
            \item Each job $j$ has a \emph{deadline} $\deadline{j}$ - a time horizon in which the job
                should be completed, otherwise it is considered \emph{tardy}. In relation to its deadline,
                each job $j$ also has a \emph{tardiness weight} $\tardinessweight{j}$. This defines a penalty accumulated
                for each unit of time the job is tardy for.
                \todo{squared penalty as the optimization goal}
        \end{itemize}

        \item The jobs are executed on \emph{resources} - machines with time-variant renewable \emph{capacities}.
        \begin{itemize}
            \item Capacity of a resource $k$ during a time period $t$ is denoted as $\capacity{k}{t}$ - the amount
                of resource $k$ available for consumption by jobs during the period $t$.
                We denote the maximum capacity of a resource $k$ throughout all time periods as $\maxcapacity{k}$.

            \item Capacities of resources are consumed by jobs during their execution. For a job $j$,
                the per-period consumption of a resource $k$ is denoted as $\consumption{j}{k}$. This describes how much
                of the resource's capacity the job is consuming each period during the job's execution.

            \item Capacities are renewable, meaning that each time period the specified amount of capacity is available,
            regardless of prior capacity consumptions.

            \item \todo{Shift modes - zero capacities} Although the definition of a resource's capacity allows it
                to be an arbitrary step-function, we assume that \wording{the only values it takes are} $0$ and $\maxcapacity{k}$.
                In addition, we assume that the function is periodical.

            \item \todo{Resource consumption Step function? It would make for an elegant definition
            of some evaluation indicators.}
        \end{itemize}

        \item The definitions presented to this point are well-established in scheduling
            literature. We extend the standard project definition with the notion of \emph{orders}.
            The set of orders \todo{nexists} $\Orders = \{ j \in \Jobs \mid \not\exists i : \precedence{j}{i} \}$
            is the set of roots of the precedence intrees, i.e. the set of all jobs for which
            no successor exists. If a job is contained in the set of orders $\Orders$, we call it an \emph{order}.

            We make a simple observation; due to the intree-structure, each (weakly) connected component
            in the precedence graph contains exactly one order. \todo{each job has a unique order - component}

            Having introduced orders, we can now specify the ranges of job deadlines and tardiness weights
            \wording{related to whether the job belongs to the orders set} $\Orders$:

            $$
            \deadline{j} \begin{cases}
                \in \N    &\dots\;   j     \in \Orders \\
                +\infty   &\dots\;   j \not\in \Orders
            \end{cases}
            \qquad
            \tardinessweight{j} \begin{cases}
                \geq 0   &\dots\;   j     \in \Orders \\
                   = 0   &\dots\;   j \not\in \Orders
            \end{cases}
            $$

    \end{itemize}

    \item \todo{This might need to be an appendix entry} Notation table, based on the notation of \cite{BRUCKER1999}.
    \begin{table}
        \centering
        \begin{tabular}{ll}
            Symbol & Definition \\
            \hline
            $n$                             & Number of jobs \\
            $\Jobs = \{ 1, \dots, n \}$     & Set of jobs \\
            $\precedence{i}{j}$ or $(i,j)$  & Precedence between jobs $i$ and $j$ \\
            $\Precedences$                  & Set of precedences \\
            $G = (\Jobs, \Precedences)$     & Precedence graph \\
            $\horizon$                      & TIme horizon \\
            $1, \dots, \horizon$            & Time periods \\
            \hline
            $p_j$                   & Processing time of job $j$ \\
            $\deadline{j}$          & Deadline for a job $j$ \\
            $\tardinessweight{j}$   & Tardiness weight for a job $j$ \\
            $\maxcapacity{k}$       & Maximum capacity of resource $k$ \\
            $\capacity{k}{t}$       & Capacity of resource $k$ in time period $t$ \\
            $\consumption{j}{k}$    & Per-period consumption of a resource $k$ by a job $j$ \\
            \hline
            $\jobstart{j}$                                      & Start time of job $j$ \\
            $\Schedule = (\jobstart{1}, \dots, \jobstart{n})$   & Schedule \\
            $\jobend{j} = \jobstart{j} + \duration{j}$          & Completion time of job $j$ \\
            $\Completions = (\jobend{1}, \dots, \jobend{n})$    & All completion times of jobs \\
            \tablenote{2}{supper-scripting a value with a schedule symbol - for example
            $\jobend{j}^{(\solutionOptimal)}$ - relates that value to the specified schedule.} \\
        \end{tabular}

        \caption{Scheduling notation}
        \label{tab:SchedulingNotation}
    \end{table}

    \item Interpretation
    \begin{itemize}
        \item \todo{No preemption = no pauses}
        \item Job precedences represent the production process of manufactured goods.
            \todo{process example}
    \end{itemize}
\end{itemize}

% ~~~~~~~~~~~~~~~~~~~~~~~~~~~~~~~~~~~~~~~~~~~~~~~~~~~~~~~~~~~~~~~~~~~~~~~~~~~~~~~~~~~~~~~~~~~~~~~~~~~~~~~~~~~
\section{Constraint programming model}

Formulation of the \ac{rcpsp} via constraint programming

\todo{squared tardiness?}
\todo{Resource constraint, is there a better way, more according to the actual implementation?}

\begin{align}
    \text{minimize}   && \sum_{j \in \Jobs} \tardinessweight{j} \tardiness{j}                                 &                      &&                                              \label{csp:objective} \\
    \text{subject to} && \jobend{i}                                                                           & \leq \jobstart{j}    && \forall \precedence{i}{j} \in \Precedences   \label{csp:precedences} \\
                      && \sum_{\substack{j \in \Jobs \\ \jobstart{j} \leq t < \jobend{j}}} \consumption{j}{k} & \leq \capacity{k}{t} && \forall t \in 1, \dots, \horizon \; \forall k \in \Resources \label{csp:capacities} \\
    \text{where}      && \multispan3{$ \Schedule = (\jobstart{1}, \dots, \jobstart{n}) \in \N^n $ \hfill}                                                                            \nonumber
\end{align}

Inequalities \eqref{csp:precedences} formulate the precedence constraints - start and end times of jobs
are according to all the precedences.
Inequalities \eqref{csp:capacities} formulate the resource capacity constraints - in every time period
the combined consumption of jobs scheduled during the period cannot exceed any of the resource's capacities.
Expression \eqref{csp:objective} is the optimization minimization objective - the weighted tardiness of jobs.

% ~~~~~~~~~~~~~~~~~~~~~~~~~~~~~~~~~~~~~~~~~~~~~~~~~~~~~~~~~~~~~~~~~~~~~~~~~~~~~~~~~~~~~~~~~~~~~~~~~~~~~~~~~~~
\section{Alternative schedule}

\begin{itemize}
    \item Suppose we have obtained an optimal solution $\solutionOptimal$ to the aforementioned problem.
        Our goal is to find a solution $\solutionImproved$ which relative to $\solutionOptimal$ improves
        the optimization goal for selected orders while not differing from $\solutionOptimal$ much.

    \item Based on the obtained solution $\solutionOptimal$ we select order(s) $\OrdersSelected \subseteq \Orders$
        for which to improve tardiness.
        We consider improvement to be any non-zero decrease in the objective function with respect to orders from $\OrdersSelected$,
        i.e.

        $$
        \sum_{o \in \OrdersSelected} \tardinessweight{o} \tardinessImproved{o} < \sum_{o \in \OrdersSelected} \tardinessweight{o} \tardiness{o}.
        $$

        Given a solution $\solutionOptimal$, for selected jobs $J \subseteq \Jobs$ and a solution $\solutionImproved$
        we compute the difference between solutions $\solutionOptimal$ and $\solutionImproved$ as

        $$
        \diff{\solutionOptimal}{\solutionImproved} \defeq \sum_{j \in J} \abs{\jobend{j}^{(\solutionOptimal)} - \jobend{j}^{(\solutionImproved)}}.
        $$

        Then, assuming the original optimization goal was to minimize $\gamma(\solutionOptimal)$, the new optimization goal
        is

        $$
        \text{minimize}\; \gamma(\solutionImproved) + \alpha \diff{\solutionOptimal}{\solutionImproved}
        $$

        for a specified constant $\alpha \geq 0$.
\end{itemize}

% ~~~~~~~~~~~~~~~~~~~~~~~~~~~~~~~~~~~~~~~~~~~~~~~~~~~~~~~~~~~~~~~~~~~~~~~~~~~~~~~~~~~~~~~~~~~~~~~~~~~~~~~~~~~
\section{Bottlenecks}

\begin{itemize}
    \item General bottleneck definition
        \begin{defn}[Bottleneck \citep{BETTERTON2012}]\label{def:bottleneck}
            The \emph{bottleneck} is the resource that affects the performance of a system in the strongest
            manner, that is, the resource that, for a given differential increment of change, has
            the largest influence on system performance.
        \end{defn}

    \item \todo{Time-dependent bottleneck definition - based on the alternative solution?}
\end{itemize}
