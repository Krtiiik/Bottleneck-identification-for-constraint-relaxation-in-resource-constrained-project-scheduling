\chapter{Problem statement}

% ~~~~~~~~~~~~~~~~~~~~~~~~~~~~~~~~~~~~~~~~~~~~~~~~~~~~~~~~~~~~~~~~~~~~~~~~~~~~~~~~~~~~~~~~~~~~~~~~~~~~~~~~~~~
\section{Scheduling}

\begin{itemize}
    \item General definition of \ac{rcpsp}

    \item Describe the $PSm \;|\; prec \;|\; \sum_{j} w_j T_j$ problem.
        Schema defined by \citet{BRUCKER1999}.
    \begin{itemize}
        \item \emph{Jobs} $1, \dots, n$.
        \begin{itemize}
            \item Each job $j$ has an execution \emph{duration} $p_j$ - the amount of time needed to process
                the job $j$. Execution of jobs is \emph{preemptive} - the execution of a job cannot be interrupted
                after its start.
            \item Execution order of jobs is constrained with \emph{precedence constraints}.
                A precedence constraint between jobs $i$ and $j$, denoted as $i \rightarrow j$,
                describes that processing of job $j$ can start only after the processing of job $i$
                was completed.
        \end{itemize}
        
        \item The project also consists of \emph{resources} - machines with constant renewable \emph{capacities}.
        \begin{itemize}
            \item Capacity of a resource $k$ is denoted as $R_k$. Capacity describes that each time period,
                resource $k$ has $R_k$ units of its capacity available for use. Capacities are renewable,
                meaning that each time period the same amount of capacity is available, regardless of prior
                capacity consumptions.
            \item Capacities of resources are consumed by jobs during their execution. For a job $j$,
                the per-period consumption of the resource $k$ is denoted as $r_{jk}$. This describes how much
                of the resource's capacity the job is consuming each period during the job's execution.
            \item \todo{Shift modes - zero capacities}
            \item \TODO{Resource consumption Step function? It would make for an elegant definition
            of some }
        \end{itemize}

        \item \TODO{Components}
    \end{itemize}

    \item Interpretation
    \begin{itemize}
        \item \todo{No preemption = no pauses}
        \item Job precedences represent the production process of manufactured goods.
            \todo{process example}
    \end{itemize}
\end{itemize}

% ~~~~~~~~~~~~~~~~~~~~~~~~~~~~~~~~~~~~~~~~~~~~~~~~~~~~~~~~~~~~~~~~~~~~~~~~~~~~~~~~~~~~~~~~~~~~~~~~~~~~~~~~~~~
\section{Constraint programming model}

\begin{itemize}
    \item Formulation of the \ac{rcpsp} via constraint programming
    \item Structured definition which enables later redefinition of relaxed constraints
\end{itemize}

% ~~~~~~~~~~~~~~~~~~~~~~~~~~~~~~~~~~~~~~~~~~~~~~~~~~~~~~~~~~~~~~~~~~~~~~~~~~~~~~~~~~~~~~~~~~~~~~~~~~~~~~~~~~~
\section{Alternative schedule}

\begin{itemize}
    \item Suppose we have obtained a solution to the aforementioned problem
    \item Select job(s) for which to improve tardiness
    \item Formulate the alternative model for which to find the alternative solution
    \item Find an alternative solution achieving improvement of the selected tardiness while
        minimizing the solution difference
\end{itemize}

% ~~~~~~~~~~~~~~~~~~~~~~~~~~~~~~~~~~~~~~~~~~~~~~~~~~~~~~~~~~~~~~~~~~~~~~~~~~~~~~~~~~~~~~~~~~~~~~~~~~~~~~~~~~~
\section{Bottlenecks}

\begin{itemize}
    \item General bottleneck definition
    \item Time-dependent bottleneck definition - based on the alternative solution?
\end{itemize}
