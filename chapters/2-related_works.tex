\chapter{Related works} \label{chap:related-works}

% ~~~~~~~~~~~~~~~~~~~~~~~~~~~~~~~~~~~~~~~~~~~~~~~~~~~~~~~~~~~~~~~~~~~~~~~~~~~~~~~~~~~~~~~~~~~~~~~~~~~~~~~~~~~
\section*{Briefly on scheduling} \label{sec:related-works/briefly-on-scheduling}

\begin{itemize}
    \item Scheduling is a broadly studied problem in literature. This is due to the practicality
        of the problem and high utilization of achieved results.
    \begin{itemize}
        \item Applying scheduling theory on a specific practical problem, case studies.
        \item Developing the scheduling theory.
        \item Applying results from other fields to problems in scheduling, e.g. mathematical programming
            \citep{KNOP2018}
    \end{itemize}

    \item \todo{}

    \item \todo{Why we use exact methods vs why are heuristical methods usually used}

\end{itemize}
    
% ~~~~~~~~~~~~~~~~~~~~~~~~~~~~~~~~~~~~~~~~~~~~~~~~~~~~~~~~~~~~~~~~~~~~~~~~~~~~~~~~~~~~~~~~~~~~~~~~~~~~~~~~~~~
\section*{Bottlenecks in scheduling} \label{sec:related-works/bottlenecks-in-scheduling}

\begin{itemize}
    \item Bottlenecks play an important role in scheduling. This statement is well captured
        in \cref{def:bottleneck}. That definition, however, leaves a lot of room for interpretation.
        \begin{itemize}
            \item How to measure affect on performance?
            \item What is a differential increment of change?
            \item Based on that, how to measure influence on system performance?
        \end{itemize}

    \item Two approaches
    \begin{itemize}
        \item Bottlenecks in creating a schedule --- identified as a heuristic to guide the search
            for (near) optimal solutions.
        \begin{itemize}
            \item \citet{ADAMS1988}, shifting bottlenecks
        \end{itemize}

        \item Bottlenecks in a presented schedule --- identified to gain information about the system's
            limitations with possible aim to improve those limitations.
        \begin{itemize}
            \item \citet{LAWRENCE1994} studied \emph{shifting bottlenecks}\footnote{The same term
                \emph{shifting bottleneck} is used for a slightly different concept, as opposed to the work
                of \citet{ADAMS1988}} in a completed schedule. The study focuses on how does
                the identification of the bottleneck machine changes through time.

                \todo{This might be the literature closest to our problem. Instead of identifying a single
                    machine as a bottleneck for the whole system during the entire scheduling horizon,
                    they proposed a mechanism to observe bottlenecks in time.}
        \end{itemize}
        
    \end{itemize}

    \item Shifting bottlenecks --- optimal schedule building based on a shifting bottleneck
    \item 
\end{itemize}

% ~~~~~~~~~~~~~~~~~~~~~~~~~~~~~~~~~~~~~~~~~~~~~~~~~~~~~~~~~~~~~~~~~~~~~~~~~~~~~~~~~~~~~~~~~~~~~~~~~~~~~~~~~~~
\section*{Bottleneck identification} \label{sec:related-works/bottleneck-identification}

Identifying 

\begin{itemize}
    \item Simple evaluation indicators
    \item Combinatorial properties
    \item Properties of an alternative optimization model
    \item Impact on scheduling
\end{itemize}

\begin{itemize}
    \item \emph{Simple evaluation indicators} are studied the most.

    \item Some indicators:
    \begin{itemize}
        \item \ac{mw} \citep{}:
            $$
            \indMW{k} = \sum_{j \;:\; \text{$j$ consumes $k$}} p_j
            $$ 
            This sums processing times of jobs which consume the resource $k$.

        \item \ac{mur}:
            $$
            \indMUR{k} = \frac{\indMW{k}}{\max_{j \;:\; \text{$j$ consumes $k$}} C_j - \min_{j \;:\; \text{$j$ consumes $k$}} S_j}
            $$

        \item \ac{auad} \citep{ROSER2001}:
            $$
            \indAUAD{k} = \frac{\sum_{a \in A_k} \abs{a}}{\abs{A_k}}
            $$
            where $A_k$ is the \todo{set of uninterrupted periods}
    \end{itemize}
\end{itemize}
