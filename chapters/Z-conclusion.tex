\chapwithtoc{Conclusion} \label{chap:conclussion}

In this thesis, we addressed the problem of reducing the tardiness of a selected manufacturing order.
First, we formulated an extension of the standard \ac{rcpsp} to model the problem.
We then focused on manufacturing bottlenecks,
specifically execution-level bottlenecks in obtained schedules.
Following the identification of such bottlenecks,
we proposed relaxations for related resource capacity constraints.

% -----------------------------------------------------------------------------------------------------------
\secwithtoc{Contribution}

The subject of identifying execution bottlenecks and subsequently relaxing related constraints
has not been studied on the variant of the \acs{rcpsp} used in this thesis to model the problem at hand.
We proposed two methods to address this problem;
the \ac{iira}, utilizing adaptations of methods
that address this problem in the Job-Shop scheduling problem,
and the \ac{ssira}, designed specifically for this problem.
We conducted numerical experiments to analyze the capabilities of the proposed methods.
For that purpose, we designed a wide variety of problem instances
modeling the extension of the \ac{rcpsp} used in this thesis.
The performances of the two methods were studied on the presented problem instances.
We observed that the \ac{ssira} is more consistent in finding improving solutions than the \ac{iira}.
However, on many instances, the \ac{iira} is able to find great improvements with low modification costs
where the \ac{ssira} finds the same improvements with greater costs.

% -----------------------------------------------------------------------------------------------------------
\secwithtoc{Further work}

The \ac{ssira} algorithm selects improvement intervals
and proposes relaxations of related resource capacity constraints.
However, it currently assumes independence among relaxations,
which does not correspond to the actual complex dependencies in the solved models.
The relaxations could alternatively be modeled as an optimization problem,
which would better capture the dependencies.

Additionally, the \ac{ssira} finds improvement intervals only for jobs
included in the left-shift closure of the target manufacturing order.
This approach has proven effective at limiting the number of jobs considered for improvement.
However, different heuristics could be used to focus the search on the targeted manufacturing order.
Alternatively, useful information could be extracted from the constraint programming solvers
used to solve the problem instance models,
as the solvers usually provide details about the solving process,
such as statistics about variable and constraint conflicts.

In the conducted experiments, we measure the schedule difference between the original schedule
and the proposed modified schedule as the total sum off job-start differences between the schedules.
This simple metric was sufficient to compare the two presented algorithms,
however, the proposed changes could be studied further to evaluate their impact on the system.
For example, by following the work of \citet{Lawrence1994}, the bottleneck \enquote{shiftiness} could be studied
to evaluate how the proposed changes affect the bottleneck identification.
