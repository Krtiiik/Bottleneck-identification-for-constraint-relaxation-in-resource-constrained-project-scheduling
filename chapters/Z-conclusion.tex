\chapwithtoc{Conclusion} \label{chap:conclussion}

In this thesis, we addressed the problem of reducing the tardiness of a selected order.
First, we formulated an extension of the standard \ac{rcpsp} to model the problem.
We then focused on manufacturing bottlenecks,
specifically execution-level bottlenecks in obtained schedules.
Following the identification of such bottlenecks,
we proposed relaxations for related resource capacity constraints.

% -----------------------------------------------------------------------------------------------------------
\secwithtoc{Contribution}

The subject of identifying execution bottlenecks and subsequently relaxing related constraints
has not been studied on the variant of the \acs{rcpsp} used in this thesis to model the problem at hand.
We proposed two methods to address this problem;
the \ac{iira}, utilizing adaptations of methods
that address this problem in the Job-Shop scheduling problem,
and the \ac{ssira}, designed specifically for this problem.
We conducted numerical experiments to analyze the capabilities of the proposed methods.
For that purpose, we designed a wide variety of problem instances
modeling the extension of the \ac{rcpsp} used in this thesis.
The performances of the two methods were studied on the presented problem instances.
We concluded that both proposed methods provide a unique approach to the problem
and neither of them can be decisively declared superior to the other.

% -----------------------------------------------------------------------------------------------------------
\secwithtoc{Further research}

Sensitivity analysis was not conducted.
Following the work of \citet{Lawrence1994},
the effect of our proposed changes could be studied
to evaluate how the changes affect the sensitivity of the system.

The idea of relaxations focusing on the target order has proven viable for reducing the order's tardiness.
The designed \ac{ssira} utilizing this approach was able to find improving solutions.
In comparison to the \enquote{naive} approach of the \ac{iira}, however,
the benefit of the focusing relaxations was not as apparent as we anticipated.
We believe this approach can be studied further to develop methods
which surpass general bottleneck-relaxing methods in performance.
