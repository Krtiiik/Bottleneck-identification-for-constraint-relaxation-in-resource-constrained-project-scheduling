%%% Please fill in basic information on your thesis, which will be automatically
%%% inserted at the right places.

% Type of your thesis:
%	"bc" for Bachelor's
%	"mgr" for Master's
%	"phd" for PhD
%	"rig" for rigorosum
\def\ThesisType{bc}

% Language of your study programme:
%	"cs" for Czech
%	"en" for English
\def\StudyLanguage{cs}

% Thesis title in English (exactly as in the official assignment)
% (Note: \xxx is a "ToDo label" which makes the unfilled visible. Remove it.)
\def\ThesisTitle{Bottleneck identification for constraint relaxation in resource-constrained project scheduling}

% Author of the thesis (you)
\def\ThesisAuthor{Lukáš Nedbálek}

% Year when the thesis is submitted
\def\YearSubmitted{2024}

% Name of the department or institute, where the work was officially assigned
% (according to the Organizational Structure of MFF UK in English,
% see https://www.mff.cuni.cz/en/faculty/organizational-structure,
% or a full name of a department outside MFF)
\def\Department{Department of Theoretical Computer Science and Mathematical Logic}

% Is it a department (katedra), or an institute (ústav)?
\def\DeptType{Department}

% Thesis supervisor: name, surname and titles
\def\Supervisor{RNDr. Jiří Švancara, Ph.D.}

% Supervisor's department (again according to Organizational structure of MFF)
\def\SupervisorsDepartment{Department of Theoretical Computer Science and Mathematical Logic}

% Study programme (does not apply to rigorosum theses)
\def\StudyProgramme{Computer Science (B0613A140006)}

% An optional dedication: you can thank whomever you wish (your supervisor,
% consultant, who provided you with tea and pizza, etc.)
\def\Dedication{%
I would like to express my gratitude to my supervisor, RNDr.~Jiří~Švancara,~Ph.D., for his kind guidance,
and to Ing.~Antonín~Novák,~Ph.D., for the numerous invaluable consultations on the subject.

I would also like to thank my family for their continuous support throughout my studies.

Finally, I wish to acknowledge my debugging ducks for their supportive presence during the writing process.
}

% Abstract (recommended length around 80-200 words; this is not a copy of your thesis assignment!)
\def\Abstract{%
In modern manufacturing systems, production planners create schedules
by iteratively obtaining proposed schedules and adjusting input parameters
to satisfy multiple, often competing, optimization goals.
The goal of this thesis is to address the problem
of reducing the tardiness of a particular order in an obtained schedule,
which is a practical problem commonly arising in production scheduling.
We do this by identifying bottlenecks in the schedule
and proposing relaxations to constraints related to the identified bottlenecks.
We develop two methods for this purpose.
The first method adapts existing approaches from the scheduling literature and proposes general relaxations.
The second method identifies potential improvements in relaxed versions of the problem
and proposes relaxations targeting the specific order.
To evaluate the performance of the methods,
we present a set of problem instances modeling the addressed problem
and conduct numerical experiments.
}

% 3 to 5 keywords (recommended) separated by \sep
% Keywords are useful for indexing and searching for the theses by topic.
\def\ThesisKeywords{%
scheduling\sep%
resource-constrained scheduling\sep%
bottlenecks\sep%
constraint relaxation
}

% If any of your metadata strings contains TeX macros, you need to provide
% a plain-text version for use in XMP metadata embedded in the output PDF file.
% If you are not sure, check the generated thesis.xmpdata file.
\def\ThesisAuthorXMP{\ThesisAuthor}
\def\ThesisTitleXMP{\ThesisTitle}
\def\ThesisKeywordsXMP{\ThesisKeywords}
\def\AbstractXMP{\Abstract}

% If your abstracts are long and do not fit in the infopage, you can make the
% fonts a bit smaller by this setting. (Also, you should try to compress your abstract more.)
% \def\InfoPageFont{}
\def\InfoPageFont{\small}  % uncomment to decrease font size

% If you are studing in a Czech programme, you also need to provide metadata in Czech:
% (in English programmes, this is not used anywhere)

\def\ThesisTitleCS{Identifikace úzkých hrdel pro relaxaci podmínek v rozvrhování projektů}
\def\DepartmentCS{Katedra teoretické informatiky a matematické logiky}
\def\DeptTypeCS{Katedra}
\def\SupervisorsDepartmentCS{Katedra teoretické informatiky a matematické logiky}
\def\StudyProgrammeCS{Informatika (B0613A140006)}

\def\ThesisKeywordsCS{%
plánování výroby\sep%
plánování výroby s omezenými zdroji\sep%
úzká hrdla\sep%
relaxování omezujících podmínek
}

\def\AbstractCS{%
Plánovači výroby často sestavují rozvrh výroby pomocí plánovacích nástrojů.
Iterovavaným získáváním návrhů na rozvrh a úpravami vstupních parametrů
se snaží vyhovět mnohým, často protichůdným, optimalizačním cílům.
Cílem této práce je zaměřit se na problém snižování zpoždění, tzv. tardiness, vybrané zakázky v obdrženém rozvrhu,
jakožto běžně řešený problém při plánování výroby.
Zaměříme se na identifikaci tzv. úzkých hrdel daných rozvrhů
za účelem relaxace omezujících podmínek souvisejících s identifikovanými úzkými hrdly.
Pro tento účel představíme dvě metody.
První metoda adaptuje existující přístupy z literatury v kombinaci s návrhy obecných relaxací podmínek.
Druhá metoda identifikuje potenciální zlepšení v relaxovaných verzích problému
a navrhuje relaxace zaměřující se na konkrétní zpožděnou zakázku.
Pro zhodnocení výkonu představených metod vytvoříme sadu instancí modelujících daný problém
a provedeme numerické experimenty.
}
